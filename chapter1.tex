

\chapter{A Tutorial Introduction}


Let us begin with a quick introduction in C. Our aim is to show the essential elements of the language in real programs, but without getting bogged down in details, rules, and exceptions.
At this point, we are not trying to be complete or even precise (save that the examples are meant to be correct).
We want to get you as quickly as possible to the point where you can write useful programs, and to do that we have to concentrate on the basics:
variables and constants, arithmetic, control flow, functions, and the rudiments of input and output.
We are intentionally leaving out of this chapter features of C that are important for writing bigger programs.
These include pointers, structures, most of C's rich set of operators, several control-flow statements, and the standard library.

This approach has its drawbacks.
Most notable is that the complete story on any particular feature is not found here, and the tutorial, by being brief, may also be misleading.
And because the examples do not use the full power of C, they are not as concise and elegant as they might be.
We have tried to minimize these effects, but be warned.
Another drawback is that later chapters will necessarily repeat some of this chapter.
We hope that the repetition will help you more than it annoys.

In any case, experienced programmers should be able to extrapolate from the material in this chapter to their own programming needs.
Beginners should supplement it by writing small, similar programs of their own.
Both groups can use it as a framework on which to hang the more detailed descriptions that begin in Chapter 2.


\section{Getting Started}

The only way to learn a new programming language is by writing programs in it.
The first program to write is the same for all languages:
\begin{lstlisting}
	Print the words
	hello, world
\end{lstlisting}
This is a big hurdle; to leap over it you have to be able to create the program text somewhere, compile it successfully, load it, run it, and find out where your output went.
With these mechanical details mastered, everything else is comparatively easy.

In C, the program to print ``hello, world'' is
\begin{lstlisting}
	#include <stdio.h>

	main()
	{
		printf("hello, world\n");
	}
\end{lstlisting}
Just how to run this program depends on the system you are using.
As a specific example, on the UNIX operating system you must create the program in a file whose name ends in ``\code{.c}'', such as \code{hello.c}, then compile it with the command
\begin{lstlisting}
	cc hello.c
\end{lstlisting}
If you haven't botched anything, such as omitting a character or misspelling something, the compilation will proceed silently, and make an executable file called \code{a.out}.
If you run \code{a.out} by typing the command
\begin{lstlisting}
	a.out
\end{lstlisting}
it will print
\begin{lstlisting}
	hello, world
\end{lstlisting}
On other systems, the rules will be different; check with a local expert.

Now, for some explanations about the program itself.
A C program, whatever its size, consists of functions and variables.
A function contains statements that specify the computing operations to be done, and variables store values used during the computation.
C functions are like the subroutines and functions in Fortran or the procedures and functions of Pascal.
Our example is a function named \code{main}.
Normally you are at liberty to give functions whatever names you like, but ``\code{main}'' is special - your program begins executing at the beginning of \code{main}.
This means that every program must have a \code{main} somewhere.

\code{main} will usually call other functions to help perform its job, some that you wrote, and others from libraries that are provided for you. The first line of the program,
\begin{lstlisting}
	#include <stdio.h>
\end{lstlisting}
tells the compiler to include information about the standard input/output library; the line appears at the beginning of many C source files.
The standard library is described in Chapter 7 and Appendix B.

One method of communicating data between functions is for the calling function to provide a list of values, called arguments, to the function it calls.
The parentheses after the function name surround the argument list.
In this example, \code{main} is defined to be a function that expects no arguments, which is indicated by the empty list \code{()}.

The statements of a function are enclosed in braces \code{{}}. The function \code{main} contains only one statement,
\begin{lstlisting}
	printf("hello, world\n");
\end{lstlisting}
A function is called by naming it, followed by a parenthesized list of arguments, so this calls the function \code{printf} with the argument \code{"hello, world\textbackslash n"}.
\code{printf} is a library function that prints output, in this case the string of characters between the quotes.

A sequence of characters in double quotes, like \code{"hello, world\textbackslash n"}, is called a character string or string constant.
For the moment our only use of character strings will be as arguments for \code{printf} and other functions.

The sequence \code{\textbackslash n} in the string is C notation for the newline character, which when printed advances the output to the left margin on the next line.
If you leave out the \code{\textbackslash n} (a worthwhile experiment), you will find that there is no line advance after the output is printed.
You must use \code{\textbackslash n} to include a newline character in the \code{printf} argument; if you try something like
\begin{lstlisting}
	printf("hello, world
	");
\end{lstlisting}
the C compiler will produce an error message
\code{printf} never supplies a newline character automatically, so several calls may be used to build up an output line in stages.
Our first program could just as well have been written

\begin{lstlisting}
	#include <stdio.h>

	main()
	{
		printf("hello, ");
		printf("world");
		printf("\n");
	}
\end{lstlisting}

to produce identical output.

Notice that \code{\textbackslash n} represents only a single character.
An escape sequence like \code{\textbackslash n} provides a general and extensible mechanism for representing hard-to-type or invisible characters.
Among the others that C provides are \code{\textbackslash t} for tab, \code{\textbackslash b} for backspace, \code{\textbackslash "} for the double quote and \code{\textbackslash\textbackslash} for the backslash itself.
There is a complete list in Section 2.3.
\newline

\begin{ExerciseList}
\Exercise Run the ``hello, world'' program on your system. Experiment with leaving out parts of the program, to see what error messages you get.
\Exercise Experiment to find out what happens when \code{prints}'s argument string contains \code{\textbackslash c}, where \code{c} is some character not listed above.
\end{ExerciseList}



\section{Variables and Arithmetic Expressions}


The next program uses the formula \code{\textdegree C=(5/9)(\textdegree F-32)} to print the following table of Fahrenheit temperatures and their centigrade or Celsius equivalents:
\begin{lstlisting}
	1 	-17
	20 	-6
	40 	4
	60 	15
	80 	26
	100 37
	120 48
	140 60
	160 71
	180 82
	200 93
	220 104
	240 115
	260 126
	280 137
	300 148
\end{lstlisting}
The program itself still consists of the definition of a single function named \code{main}.
It is longer than the one that printed \code{"hello, world"}, but not complicated.
It introduces several new ideas, including comments, declarations, variables, arithmetic expressions, loops , and formatted output.
\begin{lstlisting}
	#include <stdio.h>

	/* print Fahrenheit-Celsius table
	for fahr = 0, 20, ..., 300 */

	main()
	{
		int fahr, celsius;
		int lower, upper, step;
		lower = 0; 	 /* lower limit of temperature scale */
		upper = 300; /* upper limit */
		step = 20; 	 /* step size */
		fahr = lower;
		while (fahr <= upper) {
			celsius = 5 * (fahr-32) / 9;
			printf("%d\t%d\n", fahr, celsius);
			fahr = fahr + step;
		}
	}
\end{lstlisting}
The two lines
\begin{lstlisting}
	/* print Fahrenheit-Celsius table
	for fahr = 0, 20, ..., 300 */
\end{lstlisting}
are a comment, which in this case explains briefly what the program does.
Any characters between \code{/*} and \code{*/} are ignored by the compiler; they may be used freely to make a program easier to understand.
Comments may appear anywhere where a blank, tab or newline can.
In C, all variables must be declared before they are used, usually at the beginning of the function before any executable statements.
A declaration announces the properties of variables; it consists of a name and a list of variables, such as
\begin{lstlisting}
	int fahr, celsius;
	int lower, upper, step;
\end{lstlisting}
The type int means that the variables listed are integers; by contrast with float, which means floating point, i.e., numbers that may have a fractional part.
The range of both \code{int} and \code{float} depends on the machine you are using; 16-bits \code{int}s, which lie between -32768 and +32767, are common, as are 32-bit \code{int}s.
A float number is typically a 32-bit quantity, with at least six significant digits and magnitude generally between about $10^{-38}$ and $10^38$.

C provides several other data types besides \code{int} and \code{float}, including:
\begin{lstlisting}[basicstyle=\ttfamily\normalsize\itshape, keywordstyle=\color{black}]
	char 	character - a single byte
	short 	short integer
	long 	long integer
	double 	double-precision floating point
\end{lstlisting}
The size of these objects is also machine-dependent.
There are also arrays, structures and unions of these basic types, pointers to them, and functions that return them, all of which we will meet in due course.

Computation in the temperature conversion program begins with the assignment statements
\begin{lstlisting}
	lower = 0;
	upper = 300;
	step = 20;
\end{lstlisting}
which set the variables to their initial values. Individual statements are terminated by semicolons.

Each line of the table is computed the same way, so we use a loop that repeats once per output line; this is the purpose of the while loop
\begin{lstlisting}
	while (fahr <= upper) {
		...
	}
\end{lstlisting}

The \code{while} loop operates as follows:
The condition in parentheses is tested.
If it is true (fahr is less than or equal to upper), the body of the loop (the three statements enclosed in braces) is executed.
Then the condition is re-tested, and if true, the body is executed again.
When the test becomes false (fahr exceeds upper) the loop ends, and execution continues at the statement that follows the loop.
There are no further statements in this program, so it terminates.

The body of a while can be one or more statements enclosed in braces, as in the temperature converter, or a single statement without braces, as in
\begin{lstlisting}
	while (i < j)
		i = 2 * i;
\end{lstlisting}
In either case, we will always indent the statements controlled by the while by one tab stop (which we have shown as four spaces) so you can see at a glance which statements are inside the loop.
The indentation emphasizes the logical structure of the program.
Although C compilers do not care about how a program looks, proper indentation and spacing are critical in making programs easy for people to read.
We recommend writing only one statement per line, and using blanks around operators to clarify grouping.
The position of braces is less important, although people hold passionate beliefs.
We have chosen one of several popular styles. Pick a style that suits you, then use it consistently.

Most of the work gets done in the body of the loop.
The Celsius temperature is computed and assigned to the variable celsius by the statement
\begin{lstlisting}
	celsius = 5 * (fahr-32) / 9;
\end{lstlisting}
The reason for multiplying by \code{5} and dividing by \code{9} instead of just multiplying by \code{5/9} is that in C, as in many other languages, integer division truncates: any fractional part is discarded.
Since \code{5} and \code{9} are integers. \code{5/9} would be truncated to zero and so all the Celsius temperatures would be reported as zero.

This example also shows a bit more of how \code{printf} works.
\code{printf} is a general-purpose output formatting function, which we will describe in detail in Chapter 7.
Its first argument is a string of characters to be printed, with each \code{\%} indicating where one of the other (second, third, ...) arguments is to be substituted, and in what form it is to be printed.
For instance, \code{\%d} specifies an integer argument, so the statement
\begin{lstlisting}
	printf("%d\t%d\n", fahr, celsius);
\end{lstlisting}
causes the values of the two integers fahr and celsius to be printed, with a tab (\code{\textbackslash t}) between them.

Each \code{\%} construction in the first argument of \code{printf} is paired with the corresponding second argument, third argument, etc.; they must match up properly by number and type, or you will get wrong answers.

By the way, \code{printf} is not part of the C language; there is no input or output defined in C itself.
\code{printf} is just a useful function from the standard library of functions that are normally accessible to C programs.
The behaviour of \code{printf} is defined in the ANSI standard, however, so its properties should be the same with any compiler and library that conforms to the standard.

In order to concentrate on C itself, we don't talk much about input and output until chapter 7.
In particular, we will defer formatted input until then.
If you have to input numbers, read the discussion of the function \code{scanf} in Section 7.4. \code{scanf} is like \code{printf}, except that it reads input instead of writing output.

There are a couple of problems with the temperature conversion program.
The simpler one is that the output isn't very pretty because the numbers are not right-justified.
That's easy to fix; if we augment each \code{\%d} in the \code{printf} statement with a width, the numbers printed will be right-justified in their fields.

For instance, we might say
\begin{lstlisting}
	printf("%3d %6d\n", fahr, celsius);
\end{lstlisting}
to print the first number of each line in a field three digits wide, and the second in a field six digits wide, like this:
\begin{lstlisting}
	0 	-17
	20 	 -6
	40    4
	60   15
	80   26
	100  37
	...
\end{lstlisting}

The more serious problem is that because we have used integer arithmetic, the Celsius temperatures are not very accurate; for instance, \code{0F} is actually about \code{-17.8C}, not \code{-17}.
To get more accurate answers, we should use floating-point arithmetic instead of integer.
This requires some changes in the program. Here is the second version:
\begin{lstlisting}
	#include <stdio.h>
	/* print Fahrenheit-Celsius table
	for fahr = 0, 20, ..., 300; floating-point version */
	main()
	{
		float fahr, celsius;
		float lower, upper, step;
		lower = 0;   /* lower limit of temperatuire scale */
		upper = 300; /* upper limit */
		step = 20; 	 /* step size */
		fahr = lower;
		while (fahr <= upper) {
			celsius = (5.0/9.0) * (fahr-32.0);
			printf("%3.0f %6.1f\n", fahr, celsius);
			fahr = fahr + step;
		}
	}
\end{lstlisting}
This is much the same as before, except that fahr and celsius are declared to be \code{float} and the formula for conversion is written in a more natural way.
We were unable to use 5/9 in the previous version because integer division would truncate it to zero.
A decimal point in a constant indicates that it is floating point, however, so \code{5.0/9.0} is not truncated because it is the ratio of two floating-point values.

If an arithmetic operator has integer operands, an integer operation is performed.
If an arithmetic operator has one floating-point operand and one integer operand, however, the integer will be converted to floating point before the operation is done.
If we had written \code{(fahr-32)}, the 32 would be automatically converted to floating point.
Nevertheless, writing floating-point constants with explicit decimal points even when they have integral values emphasizes their floating-point nature for human readers.

The detailed rules for when integers are converted to floating point are in Chapter 2.
For now, notice that the assignment
\begin{lstlisting}
	fahr = lower;
\end{lstlisting}
and the test
\begin{lstlisting}
	while (fahr <= upper)
\end{lstlisting}
also work in the natural way -- the \code{int} is converted to \code{float} before the operation is done.

The \code{printf} conversion specification \code{\%3.0f} says that a floating-point number (here fahr) is to be printed at least three characters wide, with no decimal point and no fraction digits.
\code{\%6.1f} describes another number (celsius) that is to be printed at least six characters wide, with \code{1} digit after the decimal point.
The output looks like this:

\begin{lstlisting}
	0 	-17.8
	20 	 -6.7
	40 	  4.4
	...
\end{lstlisting}

Width and precision may be omitted from a specification:
\code{\%6f} says that the number is to be at least six characters wide; \code{\%.2f} specifies two characters after the decimal point, but the width is not constrained; and \code{\%f} merely says to print the number as floating point.

Among others, \code{printf} also recognizes \code{\%o} for octal, \code{\%x} for hexadecimal, \code{\%c} for character, \code{\%s} for character string and \code{\%\%} for itself
\newline

\begin{ExerciseList}
\Exercise Modify the temperature conversion program to print a heading above the table.
\Exercise Write a program to print the corresponding Celsius to Fahrenheit table.
\end{ExerciseList}



\section{The for statement}


There are plenty of different ways to write a program for a particular task. Let's try a variation on the temperature converter.
\begin{lstlisting}
	#include <stdio.h>

	/* print Fahrenheit-Celsius table */
	main()
	{
		int fahr;
		for (fahr = 0; fahr <= 300; fahr = fahr + 20)
			printf("%3d %6.1f\n", fahr, (5.0/9.0)*(fahr-32));
	}

\end{lstlisting}
This produces the same answers, but it certainly looks different.
One major change is the elimination of most of the variables; only fahr remains, and we have made it an \code{int}.
The lower and upper limits and the step size appear only as constants in the \code{for} statement, itself a new construction, and the expression that computes the Celsius temperature now appears as the third argument of printf instead of a separate assignment statement.

This last change is an instance of a general rule -- in any context where it is permissible to use the value of some type, you can use a more complicated expression of that type.
Since the third argument of \code{printf} must be a floating-point value to match the \code{\%6.1f}, any floating-point expression can occur here.

The \code{for} statement is a loop, a generalization of the \code{while}.
If you compare it to the earlier \code{while}, its operation should be clear.
Within the parentheses, there are three parts, separated by semicolons.
The first part, the initialization
\begin{lstlisting}
	fahr = 0
\end{lstlisting}
is done once, before the loop proper is entered. The second part is the test or condition that controls the loop:
\begin{lstlisting}
	fahr <= 300
\end{lstlisting}
This condition is evaluated; if it is true, the body of the loop (here a single \code{printf}) is executed. Then the increment step
\begin{lstlisting}
	fahr = fahr + 20
\end{lstlisting}
is executed, and the condition re-evaluated.
The loop terminates if the condition has become false.
As with the \code{while}, the body of the loop can be a single statement or a group of statements enclosed in braces.
The initialization, condition and increment can be any expressions.

The choice between \code{while} and \code{for} is arbitrary, based on which seems clearer.
The \code{for} is usually appropriate \code{for} loops in which the initialization and increment are single statements and logically related, since it is more compact than \code{while} and it keeps the loop control statements together in one place.
\newline

\begin{ExerciseList}
\Exercise Modify the temperature conversion program to print the table in reverse order, that is, from 300 degrees to 0.
\end{ExerciseList}



\section{Symbolic Constants}


A final observation before we leave temperature conversion forever.
It's bad practice to bury ``magic numbers'' like 300 and 20 in a program; they convey little information to someone who might have to read the program later, and they are hard to change in a systematic way.

One way to deal with magic numbers is to give them meaningful names.
A \code{\#define} line defines a symbolic name or symbolic constant to be a particular string of characters:
\begin{lstlisting}[basicstyle=\ttfamily\normalsize\itshape, keywordstyle=\color{black}]
	#define name replacement list
\end{lstlisting}
Thereafter, any occurrence of name (not in quotes and not part of another name) will be replaced by the corresponding replacement text.
The name has the same form as a variable name: a sequence of letters and digits that begins with a letter.
The replacement text can be any sequence of characters; it is not limited to numbers.
\begin{lstlisting}
	#include <stdio.h>

	#define LOWER 0 /* lower limit of table */
	#define UPPER 300 /* upper limit */
	#define STEP 20 /* step size */

	/* print Fahrenheit-Celsius table */
	main()
	{
		int fahr;
		for (fahr = LOWER; fahr <= UPPER; fahr = fahr + STEP)
			printf("%3d %6.1f\n", fahr, (5.0/9.0)*(fahr-32));
	}
\end{lstlisting}

The quantities \code{LOWER}, \code{UPPER} and \code{STEP} are symbolic constants, not variables, so they do not appear in declarations.
Symbolic constant names are conventionally written in upper case so they can ber readily distinguished from lower case variable names.
Notice that there is no semicolon at the end of a \code{\#define} line.



\section{Character Input and Output}


We are going to consider a family of related programs for processing character data.
You will find that many programs are just expanded versions of the prototypes that we discuss here.

The model of input and output supported by the standard library is very simple.
Text input or output, regardless of where it originates or where it goes to, is dealt with as streams of characters.
A text stream is a sequence of characters divided into lines; each line consists of zero or more characters followed by a newline character.
It is the responsibility of the library to make each input or output stream confirm this model; the C programmer using the library need not worry about how lines are represented outside the program.

The standard library provides several functions for reading or writing one character at a time, of which \code{getchar} and \code{putchar} are the simplest.
Each time it is called, \code{getchar} reads the next input character from a text stream and returns that as its value.
That is, after \code{c = getchar();} the variable \code{c} contains the next character of input.
The characters normally come from the keyboard; input from files is discussed in Chapter 7.
The function \code{putchar} prints a character each time it is called: \code{putchar(c);} prints the contents of the integer variable c as a character, usually on the screen.
Calls to \code{putchar} and \code{printf} may be interleaved; the output will appear in the order in which the calls are made.


\subsection{File Copying}

Given \code{getchar} and \code{putchar}, you can write a surprising amount of useful code without knowing anything more about input and output.
The simplest example is a program that copies its input to its output one character at a time:
\begin{lstlisting}[basicstyle=\ttfamily\normalsize\itshape, keywordstyle=\color{black}]
	read a character
	while (charater is not end-of-file indicator)
		output the character just read
		read a character
\end{lstlisting}
Converting this into C gives:
\begin{lstlisting}
	#include <stdio.h>
	/* copy input to output; 1st version */
	main()
	{
		int c;
		c = getchar();
		while (c != EOF) {
			putchar(c);
			c = getchar();
		}
	}
\end{lstlisting}
The relational operator \code{!=} means ``not equal to''.

What appears to be a character on the keyboard or screen is of course, like everything else, stored internally just as a bit pattern.
The type char is specifically meant for storing such character data, but any integer type can be used.
We used int for a subtle but important reason.

The problem is distinguishing the end of input from valid data.
The solution is that \code{getchar} returns a distinctive value when there is no more input, a value that cannot be confused with any real character.
This value is called \code{EOF}, for ``end of file''.
We must declare \code{c} to be a type big enough to hold any value that \code{getchar} returns.
We can't use char since \code{c} must be big enough to hold \code{EOF} in addition to any possible char.
Therefore we use int.

\code{EOF} is an integer defined in \code{<stdio.h>}, but the specific numeric value doesn't matter as long as it is not the same as any char value.
By using the symbolic constant, we are assured that nothing in the program depends on the specific numeric value.

The program for copying would be written more concisely by experienced C programmers.
In C, any assignment, such as
\begin{lstlisting}
	c = getchar();
\end{lstlisting}
is an expression and has a value, which is the value of the left hand side after the assignment.
This means that a assignment can appear as part of a larger expression.
If the assignment of a character to \code{c} is put inside the test part of a while loop, the copy program can be written this way:
\begin{lstlisting}
	#include <stdio.h>

	/* copy input to output; 2nd version */
	main()
	{
		int c;
		while ((c = getchar()) != EOF)
			putchar(c);
	}
\end{lstlisting}
The while gets a character, assigns it to \code{c}, and then tests whether the character was the end-of-file signal.
If it was not, the body of the while is executed, printing the character.
The \code{while} then repeats.
When the end of the input is finally reached, the \code{while} terminates and so does \code{main}.

This version centralizes the input -- there is now only one reference to \code{getchar} -- and shrinks the program.
The resulting program is more compact, and, once the idiom is mastered, easier to read. You'll see this style often.
(It's possible to get carried away and create impenetrable code, however, a tendency that we will try to curb.)

The parentheses around the assignment, within the condition are necessary.
The precedence of \code{!=} is higher than that of \code{=}, which means that in the absence of parentheses the relational test \code{!=} would be done before the assignment \code{=}.
So the statement
\begin{lstlisting}
	c = getchar() != EOF
\end{lstlisting}
is equivalent to
\begin{lstlisting}
	c = (getchar() != EOF)
\end{lstlisting}
This has the undesired effect of setting \code{c} to \code{0} or \code{1}, depending on whether or not the call of getchar returned end of file. (More on this in Chapter 2.)
\newline

\begin{ExerciseList}
\Exercise Verify that the expression \code{getchar() != EOF} is \code{0} or \code{1}.
\Exercise Write a program to print the value of \code{EOF}.
\end{ExerciseList}



\subsection{Character Counting}


The next program counts characters; it is similar to the copy program.
\begin{lstlisting}
	#include <stdio.h>

	/* count characters in input; 1st version */
	main()
	{
		long nc;
		nc = 0;
		while (getchar() != EOF)
			++nc;
		printf("%ld\n", nc);
	}
\end{lstlisting}
The statement
\begin{lstlisting}
	++nc;
\end{lstlisting}
presents a new operator, \code{++}, which means \emph{increment by one}.
You could instead write \code{nc = nc + 1} but \code{++nc} is more concise and often more efficient.
There is a corresponding operator \code{--} to decrement by 1.
The operators \code{++} and \code{--} can be either prefix operators (\code{++nc}) or postfix operators (\code{nc++});
these two forms have different values in expressions, as will be shown in Chapter 2, but \code{++nc} and \code{nc++} both increment \code{nc}.
For the moment we will will stick to the prefix form.

The character counting program accumulates its count in a long variable instead of an \code{int}.
\code{long} integers are at least 32 bits.
Although on some machines, \code{int} and \code{long} are the same size, on others an \code{int} is 16 bits, with a maximum value of 32767, and it would take relatively little input to overflow an \code{int} counter.
The conversion specification \code{\%ld} tells \code{printf} that the corresponding argument is a \code{long} integer.

It may be possible to cope with even bigger numbers by using a \code{double} (double precision \code{float}).
We will also use a \code{for} statement instead of a \code{while}, to illustrate another way to write the loop.
\begin{lstlisting}
	#include <stdio.h>

	/* count characters in input; 2nd version */
	main()
	{
		double nc;
		for (nc = 0; getchar() != EOF; ++nc)
			;
		printf("%.0f\n", nc);
	}
\end{lstlisting}
\code{printf} uses \code{\%f} for both \code{float} and \code{double}; \code{\%.0f} suppresses the printing of the decimal point and the fraction part, which is zero.

The body of this for loop is empty, because all the work is done in the test and increment parts.
But the grammatical rules of C require that a \code{for} statement have a body.
The isolated semicolon, called a \code{null statement}, is there to satisfy that requirement.
We put it on a separate line to make it visible.

Before we leave the character counting program, observe that if the input contains no characters, the \code{while} or \code{for} test fails on the very first call to \code{getchar}, and the program produces zero, the right answer.
This is important.
One of the nice things about \code{while} and \code{for} is that they test at the top of the loop, before proceeding with the body.
If there is nothing to do, nothing is done, even if that means never going through the loop body.
Programs should act intelligently when given zero-length input.
The \code{while} and \code{for} statements help ensure that programs do reasonable things with boundary conditions.



\subsection{Line Counting}


The next program counts input lines.
As we mentioned above, the standard library ensures that an input text stream appears as a sequence of lines, each terminated by a newline.
Hence, counting lines is just counting newlines:
\begin{lstlisting}
	#include <stdio.h>

	/* count lines in input */
	main()
	{
		int c, nl;
		nl = 0;
		while ((c = getchar()) != EOF)
			if (c == '\n')
				++nl;
		printf("%d\n", nl);
	}
\end{lstlisting}
The body of the \code{while} now consists of an \code{if}, which in turn controls the increment \code{++nl}.
The \code{if} statement tests the parenthesized condition, and \code{if} the condition is true, executes the statement (or group of statements in braces) that follows.
We have again indented to show what is controlled by what.

The double equals sign \code{==} is the C notation for ``is equal to'' (like Pascal's single \code{=} or Fortran's \code{.EQ.}).
This symbol is used to distinguish the equality test from the single \code{=} that C uses for assignment.
A word of caution: newcomers to C occasionally write \code{=} when they mean \code{==}.
As we will see in Chapter 2, the result is usually a legal expression, so you will get no warning.

A character written between single quotes represents an integer value equal to the numerical value of the character in the machine's character set.
This is called a \emph{character constant}, although it is just another way to write a small integer.
So, for example, \code{'A'} is a character constant; in the ASCII character set its value is 65, the internal representation of the character \textsc{A}.
Of course, \code{'A'} is to be preferred over 65: its meaning is obvious, and it is independent of a particular character set.

The escape sequences used in string constants are also legal in character constants, so \code{'\textbackslash n'} stands for the value of the newline character, which is 10 in ASCII.
You should note carefully that \code{'\textbackslash n'} is a single character, and in expressions is just an integer; on the other hand, \code{'\textbackslash n'} is a string constant that happens to contain only one character.
The topic of strings versus characters is discussed further in Chapter 2.
\newline

\begin{ExerciseList}
\Exercise Write a program to count blanks, tabs, and newlines.
\Exercise Write a program to copy its input to its output, replacing each string of one or more blanks by a single blank.
\Exercise Write a program to copy its input to its output, replacing each tab by \code{\textbackslash t}, each backspace by \code{\textbackslash b}, and each backslash by \code{\textbackslash\textbackslash}. This makes tabs and backspaces visible in an unambiguous way.
\end{ExerciseList}


\subsection{Word Counting}


The fourth in our series of useful programs counts lines, words, and characters, with the loose definition that a word is any sequence of characters that does not contain a blank, tab or newline.
This is a bare-bones version of the UNIX program \code{wc}.
\begin{lstlisting}
	#include <stdio.h>
	#define IN 1 /* inside a word */
	#define OUT 0 /* outside a word */

	/* count lines, words, and characters in input */
	main()
	{
		int c, nl, nw, nc, state;
		state = OUT;
		nl = nw = nc = 0;
		while ((c = getchar()) != EOF) {
			++nc;
			if (c == '\n')
				++nl;
			if (c == ' ' || c == '\n' || c == '\t')
				state = OUT;
			else if (state == OUT) {
				state = IN;
				++nw;
			}
		}
		printf("%d %d %d\n", nl, nw, nc);
	}
\end{lstlisting}
Every time the program encounters the first character of a word, it counts one more word.
The variable state records whether the program is currently in a word or not; initially it is ``not in a word'', which is assigned the value \code{OUT}.
We prefer the symbolic constants \code{IN} and \code{OUT} to the literal values \code{1} and \code{0} because they make the program more readable.
In a program as tiny as this, it makes little difference, but in larger programs, the increase in clarity is well worth the modest extra effort to write it this way from the beginning.
You'll also find that it's easier to make extensive changes in programs where magic numbers appear only as symbolic constants.

The line
\begin{lstlisting}
	nl = nw = nc = 0;
\end{lstlisting}
sets all three variables to zero.
This is not a special case, but a consequence of the fact that an assignment is an expression with the value and assignments associated from right to left.
It's as if we had written
\begin{lstlisting}
	nl = (nw = (nc = 0));
\end{lstlisting}
The operator \code{||} means OR, so the line
\begin{lstlisting}
	if (c == ' ' || c == '\n' || c = '\t')
\end{lstlisting}
says ``if \code{c} is a blank or \code{c} is a newline or \code{c} is a tab''.
(Recall that the escape sequence \code{\textbackslash t} is a visible representation of the tab character.)
There is a corresponding operator \code{\&\&} for AND; its precedence is just higher than \code{||}.
Expressions connected by \code{\&\&} or \code{||} are evaluated left to right, and it is guaranteed that evaluation will stop as soon as the truth or falsehood is known.
If \code{c} is a blank, there is no need to test whether it is a newline or tab, so these tests are not made.
This isn't particularly important here, but is significant in more complicated situations, as we will soon see.
The example also shows an else, which specifies an alternative action if the condition part of an if statement is false.
The general form is
\begin{lstlisting}[basicstyle=\ttfamily\normalsize\itshape, keywordstyle=\color{black}]
	if (expression)
		statement1
	else
		statement2
\end{lstlisting}
One and only one of the two statements associated with an \code{if-else} is performed.
If the expression is true, statement1 is executed; if not, statement2 is executed.
Each statement can be a single statement or several in braces.
In the word count program, the one after the \code{else} is an \code{if} that controls two statements in braces.
\newline

\begin{ExerciseList}
\Exercise How would you test the word count program? What kinds of input are most likely to uncover bugs if there are any?
\Exercise Write a program that prints its input one word per line.
\end{ExerciseList}


\section{Arrays}


Let us write a program to count the number of occurrences of each digit, of white space characters (blank, tab, newline), and of all other characters.
This is artificial, but it permits us to illustrate several aspects of C in one program.
There are twelve categories of input, so it is convenient to use an array to hold the number of occurrences of each digit, rather than ten individual variables.
Here is one version of the program:
\begin{lstlisting}
	#include <stdio.h>

	/* count digits, white space, others */
	main()
	{
		int c, i, nwhite, nother;
		int ndigit[10];

		nwhite = nother = 0;
		for (i = 0; i < 10; ++i)
			ndigit[i] = 0;
		while ((c = getchar()) != EOF)
			if (c >= '0' && c <= '9')
				++ndigit[c-'0'];
		else if (c == ' ' || c == '\n' || c == '\t')
			++nwhite;
		else
			++nother;

		printf("digits =");
		for (i = 0; i < 10; ++i)
			printf(" %d", ndigit[i]);
		printf(", white space = %d, other = %d\n", nwhite, nother);
	}
\end{lstlisting}
The output of this program on itself is
\begin{lstlisting}
	digits = 9 3 0 0 0 0 0 0 0 1, white space = 123, other = 345
\end{lstlisting}

The declaration
\begin{lstlisting}
	int ndigit[10];
\end{lstlisting}
declares \code{ndigit} to be an array of \code{10} integers.
Array subscripts always start at zero in C, so the elements are \code{ndigit[0]}, \code{ndigit[1]}, ..., \code{ndigit[9]}.
This is reflected in the for loops that initialize and print the array.

A subscript can be any integer expression, which includes integer variables like \code{i}, and integer constants.

This particular program relies on the properties of the character representation of the digits.
For example, the test
\begin{lstlisting}
	if (c >= '0' && c <= '9')
\end{lstlisting}
determines whether the character in \code{c} is a digit.
If it is, the numeric value of that digit is
\begin{lstlisting}
	c - '0'
\end{lstlisting}
This works only if \code{'0'}, \code{'1'}, ..., \code{'9'} have consecutive increasing values.
Fortunately, this is true for all character sets.

By definition, \code{char}s are just small integers, so \code{char} variables and constants are identical to \code{int}s in arithmetic expressions.
This is natural and convenient; for example \code{c - '0'} is an integer expression with a value between 0 and 9 corresponding to the character \code{'0'} to \code{'9'} stored in \code{c}, and thus a valid subscript for the array \code{ndigit}.

The decision as to whether a character is a digit, white space, or something else is made with the sequence
\begin{lstlisting}
	if (c >= '0' && c <= '9')
		++ndigit[c-'0'];
	else if (c == ' ' || c == '\n' || c == '\t')
		++nwhite;
	else
		++nother;
\end{lstlisting}
The pattern
\begin{lstlisting}[basicstyle=\ttfamily\normalsize\itshape, keywordstyle=\color{black}]
	if (condition1)
		statement1
	else if (condition2)
		statement2
	...
	...
	else
		statementn
\end{lstlisting}
occurs frequently in programs as a way to express a multi-way decision.

The conditions are evaluated in order from the top until some condition is satisfied; at that point the corresponding statement part is executed, and the entire construction is finished.
(Any statement can be several statements enclosed in braces.)
If none of the conditions is satisfied, the statement after the final else is executed if it is present.
If the final else and statement are omitted, as in the word count program, no action takes place.
There can be any number of
\begin{lstlisting}[basicstyle=\ttfamily\normalsize\itshape, keywordstyle=\color{black}]
	else if(condition)
		statement
\end{lstlisting}
groups between the initial \code{if} and the final \code{else}.

As a matter of style, it is advisable to format this construction as we have shown; if each \code{if} were indented past the previous \code{else}, a long sequence of decisions would march off the right side of the page.
The \code{switch} statement, to be discussed in Chapter 4, provides another way to write a multiway branch that is particulary suitable when the condition is whether some integer or character expression matches one of a set of constants.
For contrast, we will present a \code{switch} version of this program in Section 3.4.
\newline

\begin{ExerciseList}
\Exercise Write a program to print a histogram of the lengths of words in its input. It is easy to draw the histogram with the bars horizontal; a vertical orientation is more challenging.
\Exercise Write a program to print a histogram of the frequencies of different characters in its input.
\end{ExerciseList}



\section{Functions}


In C, a function is equivalent to a subroutine or function in Fortran, or a procedure or function in Pascal.
A function provides a convenient way to encapsulate some computation, which can then be used without worrying about its implementation.
With properly designed functions, it is possible to ignore how a job is done; knowing what is done is sufficient.
C makes the use of functions easy, convinient and efficient; you will often see a short function defined and called only once, just because it clarifies some piece of code.

So far we have used only functions like \code{printf}, \code{getchar} and \code{putchar} that have been provided for us; now it's time to write a few of our own.
Since C has no exponentiation operator like the \code{**} of Fortran, let us illustrate the mechanics of function definition by writing a function \code{power(m,n)} to raise an integer \code{m} to a positive integer power \code{n}.
That is, the value of \code{power(2,5)} is \code{32}.
This function is not a practical exponentiation routine, since it handles only positive powers of small integers, but it's good enough for illustration. (The standard library contains a function \code{pow(x,y)} that computes \code{x ** y}.)

Here is the function \code{power} and a \code{main} program to exercise it, so you can see the whole structure at once.
\begin{lstlisting}
	#include <stdio.h>

	int power(int m, int n);

	/* test power function */
	main()
	{
		int i;
		for (i = 0; i < 10; ++i)
			printf("%d %d %d\n", i, power(2,i), power(-3,i));
		return 0;
	}

	/* power: raise base to n-th power; n >= 0 */
	int power(int base, int n)
	{
		int i, p;
		p = 1;
		for (i = 1; i <= n; ++i)
			p = p * base;
		return p;
	}
\end{lstlisting}

A function definition has this form:
\begin{lstlisting}[basicstyle=\ttfamily\normalsize\itshape, keywordstyle=\color{black}]
	return-type function-name(parameter declarations, if any)
	{
		declarations
		statements
	}
\end{lstlisting}
Function definitions can appear in any order, and in one source file or several, although no function can be split between files.
If the source program appears in several files, you may have to say more to compile and load it than if it all appears in one, but that is an operating system matter, not a language attribute.
For the moment, we will assume that both functions are in the same file, so whatever you have learned about running C programs will still work.

The function \code{power} is called twice by main, in the line
\begin{lstlisting}
	printf("%d %d %d\n", i, power(2,i), power(-3,i));
\end{lstlisting}
Each call passes two arguments to power, which each time returns an integer to be formatted and printed.
In an expression, \code{power(2,i)} is an integer just as \code{2} and \code{i} are.  (Not all functions produce an integer value; we will take this up in Chapter 4.)

The first line of power itself,
\begin{lstlisting}
	int power(int base, int n)
\end{lstlisting}
declares the parameter types and names, and the type of the result that the function returns.

The names used by power for its parameters are local to power, and are not visible to any other function: other routines can use the same names without conflict.
This is also true of the variables \code{i} and \code{p}: the \code{i} in power is unrelated to the \code{i} in \code{main}.

We will generally use \emph{parameter} for a variable named in the parenthesized list in a function.
The terms \emph{formal argument} and \emph{actual argument} are sometimes used for the same distinction.

The value that power computes is returned to \code{main} by the \code{return} statement.
Any expression may follow return:
\begin{lstlisting}[basicstyle=\ttfamily\normalsize\itshape, keywordstyle=\color{black}]
	return expression;
\end{lstlisting}
A function need not return a value; a return statement with no expression causes control, but no useful value, to be returned to the caller, as does ``falling off the end'' of a function by reaching the terminating right brace.
And the calling function can ignore a value returned by a function.

You may have noticed that there is a \code{return} statement at the end of \code{main}.
Since \code{main} is a function like any other, it may return a value to its caller, which is in effect the environment in which the program was executed.
Typically, a return value of zero implies normal termination; non-zero values signal unusual or erroneous termination conditions.

In the interests of simplicity, we have omitted \code{return} statements from our \code{main} functions up to this point, but we will include them hereafter, as a reminder that programs should return status to their environment.

The declaration
\begin{lstlisting}
	int power(int base, int n);
\end{lstlisting}
just before \code{main} says that \code{power} is a function that expects two int arguments and returns an \code{int}.

This declaration, which is called a \emph{function prototype}, has to agree with the definition and uses of \code{power}.
It is an error if the definition of a function or any uses of it do not agree with its prototype.

Parameter names need not agree.
Indeed, parameter names are optional in a function prototype, so for the prototype we could have written
\begin{lstlisting}
	int power(int, int);
\end{lstlisting}
Well-chosen names are good documentation however, so we will often use them.

A note of history: the biggest change between ANSI C and earlier versions is how functions are declared and defined.
In the original definition of C, the power function would have been written like this:

\begin{lstlisting}
	/* power: raise base to n-th power; n >= 0 */
	/* (old-style version) */
	power(base, n)
	int base, n;
	{
		int i, p;
		p = 1;
		for (i = 1; i <= n; ++i)
			p = p * base;
		return p;
	}
\end{lstlisting}

The parameters are named between the parentheses, and their types are declared before opening the left brace; undeclared parameters are taken as \code{int}. (The body of the function is the same as before.)
The declaration of power at the beginning of the program would have looked like this:
\begin{lstlisting}
	int power();
\end{lstlisting}
No parameter list was permitted, so the compiler could not readily check that \code{power} was being called correctly.
Indeed, since by default \code{power} would have been assumed to return an \code{int}, the entire declaration might well have been omitted.
The new syntax of function prototypes makes it much easier for a compiler to detect errors in the number of arguments or their types.
The old style of declaration and definition still works in ANSI C, at least for a transition period, but we strongly recommend that you use the new form when you have a compiler that supports it.
\newline

\begin{ExerciseList}
\Exercise Rewrite the temperature conversion program of Section 1.2 to use a function for conversion.
\end{ExerciseList}

\section{Arguments -- Call by Value}

One aspect of C functions may be unfamiliar to programmers who are used to some other languages, particulary Fortran.
In C, all function arguments are passed ``by value''. This means that the called function is given the values of its arguments in temporary variables rather than the originals.
This leads to some different properties than are seen with ``call by reference'' languages like Fortran or with var parameters in Pascal, in which the called routine has access to the original argument, not a local copy.

Call by value is an asset, however, not a liability.
It usually leads to more compact programs with fewer extraneous variables, because parameters can be treated as conveniently initialized local variables in the called routine.
For example, here is a version of power that makes use of this property.
\begin{lstlisting}
	/* power: raise base to n-th power; n >= 0; version 2 */
	int power(int base, int n)
	{
		int p;
		for (p = 1; n > 0; --n)
			p = p * base;
		return p;
	}
\end{lstlisting}
The parameter \code{n} is used as a temporary variable, and is counted down (a for loop that runs backwards) until it becomes zero; there is no longer a need for the variable \code{i}.
Whatever is done to \code{n} inside power has no effect on the argument that power was originally called with.

When necessary, it is possible to arrange for a function to modify a variable in a calling routine.
The caller must provide the address of the variable to be set (technically a pointer to the variable), and the called function must declare the parameter to be a pointer and access the variable indirectly through it.
We will cover pointers in Chapter 5.

The story is different for arrays.
When the name of an array is used as an argument, the value passed to the function is the location or address of the beginning of the array -- there is no copying of array elements.
By subscripting this value, the function can access and alter any argument of the array.
This is the topic of the next section.



\section{Character Arrays}


The most common type of array in C is the array of characters.
To illustrate the use of character arrays and functions to manipulate them, let's write a program that reads a set of text lines and prints the longest.
The outline is simple enough:
\begin{lstlisting}[basicstyle=\ttfamily\normalsize\itshape, keywordstyle=\color{black}, stringstyle=\color{black}]
	while (there's another line)
		if (it's longer than the previous longest)
			(save it)
			(save its length)
	print longest line
\end{lstlisting}
This outline makes it clear that the program divides naturally into pieces.
One piece gets a new line, another saves it, and the rest controls the process.

Since things divide so nicely, it would be well to write them that way too.
Accordingly, let us first write a separate function \code{getline} to fetch the next line of input.
We will try to make the function useful in other contexts.
At the minimum, \code{getline} has to return a signal about possible end of file; a more useful design would be to return the length of the line, or zero if end of file is encountered.
Zero is an acceptable end-of-file return because it is never a valid line length.
Every text line has at least one character; even a line containing only a newline has length 1.

When we find a line that is longer than the previous longest line, it must be saved somewhere.
This suggests a second function, copy, to copy the new line to a safe place.
Finally, we need a \code{main} program to control \code{getline} and copy. Here is the result.
\begin{lstlisting}

	#include <stdio.h>
	#define MAXLINE 1000 /* maximum input line length */

	int getline(char line[], int maxline);
	void copy(char to[], char from[]);

	/* print the longest input line */
	main()
	{
		int len; 			       /* current line length */
		int max; 			/* maximum length seen so far */
		char line[MAXLINE];         /* current input line */
		char longest[MAXLINE]; /* longest line saved here */
		max = 0;
		while ((len = getline(line, MAXLINE)) > 0)
			if (len > max) {
			max = len;
			copy(longest, line);
			}
		if (max > 0) 	/* there was a line */
			printf("%s", longest);
		return 0;
	}

	/* getline: read a line into s, return length */
	int getline(char s[],int lim)
	{
		int c, i;
		for (i=0; i < lim-1 && (c=getchar())!=EOF && c!='\n'; ++i)
		s[i] = c;
		if (c == '\n') {
			s[i] = c;
			++i;
		}
		s[i] = '\0';
		return i;
	}

	/* copy: copy from into to; assume to is big enough */
	void copy(char to[], char from[])
	{
		int i;
		i = 0;
		while ((to[i] = from[i]) != '\0')
			++i;
	}
\end{lstlisting}
The functions \code{getline} and copy are declared at the beginning of the program, which we assume is contained in one file.
\code{main} and \code{getline} communicate through a pair of arguments and a returned value.
In \code{getline}, the arguments are declared by the line
\begin{lstlisting}
	int getline(char s[], int lim);
\end{lstlisting}
which specifies that the first argument, \code{s}, is an array, and the second, \code{lim}, is an integer.

The purpose of supplying the size of an array in a declaration is to set aside storage.
The length of an array \code{s} is not necessary in \code{getline} since its size is set in \code{main}.
\code{getline} uses return to send a value back to the caller, just as the function \code{power} did.
This line also declares that \code{getline} returns an \code{int}; since \code{int} is the default return type, it could be omitted.

Some functions return a useful value; others, like \code{copy}, are used only for their effect and return no value.
The return type of \code{copy} is \code{void}, which states explicitly that no value is returned.
\code{getline} puts the character \code{'\textbackslash 0'} (the null character, whose value is zero) at the end of the array it is creating, to mark the end of the string of characters.
This conversion is also used by the C language: when a string constant like
\begin{lstlisting}
	"hello\n"
\end{lstlisting}
appears in a C program, it is stored as an array of characters containing the characters in the string and terminated with a \code{'\textbackslash 0'} to mark the end.

The \code{\%s} format specification in \code{printf} expects the corresponding argument to be a string represented in this form.
\code{copy} also relies on the fact that its input argument is terminated with a \code{'\textbackslash 0'}, and copies this character into the output.

It is worth mentioning in passing that even a program as small as this one presents some sticky design problems.
For example, what should \code{main} do if it encounters a line which is bigger than its limit?
\code{getline} works safely, in that it stops collecting when the array is full, even if no newline has been seen.
By testing the length and the last character returned, \code{main} can determine whether the line was too long, and then cope as it wishes.
In the interests of brevity, we have ignored this issue.

There is no way for a user of \code{getline} to know in advance how long an input line might be, so \code{getline} checks for overflow.
On the other hand, the user of \code{copy} already knows (or can find out) how big the strings are, so we have chosen not to add error checking to it.
\newline

\begin{ExerciseList}
\Exercise Revise the \code{main} routine of the longest-line program so it will correctly print the length of arbitrary long input lines, and as much as possible of the text.
\Exercise Write a program to print all input lines that are longer than 80 characters.
\Exercise Write a program to remove trailing blanks and tabs from each line of input, and to delete entirely blank lines.
\Exercise Write a function \code{reverse(s)} that reverses the character string \code{s}. Use it to write a program that reverses its input a line at a time.
\end{ExerciseList}



\section{External Variables and Scope}


The variables in \code{main}, such as line, longest, etc., are private or local to \code{main}.
Because they are declared within \code{main}, no other function can have direct access to them.
The same is true of the variables in other functions; for example, the variable \code{i} in \code{getline} is unrelated to the \code{i} in \code{copy}.
Each local variable in a function comes into existence only when the function is called, and disappears when the function is exited.
This is why such variables are usually known as automatic variables, following terminology in other languages.
We will use the term \emph{automatic} henceforth to refer to these local variables.
(Chapter 4 discusses the \code{static} storage class, in which local variables do retain their values between calls.)

Because automatic variables come and go with function invocation, they do not retain their values from one call to the next, and must be explicitly set upon each entry.
If they are not set, they will contain garbage.

As an alternative to automatic variables, it is possible to define variables that are external to all functions, that is, variables that can be accessed by name by any function.
(This mechanism is rather like Fortran \code{COMMON} or Pascal variables declared in the outermost block.)
Because external variables are globally accessible, they can be used instead of argument lists to communicate data between functions.
Furthermore, because external variables remain in existence permanently, rather than appearing and disappearing as functions are called and exited, they retain their values even after the functions that set them have returned.

An external variable must be \code{defined}, exactly once, outside of any function; this sets aside storage for it.
The variable must also be \code{declared} in each function that wants to access it; this states the type of the variable.
The declaration may be an explicit extern statement or may be implicit from context.
To make the discussion concrete, let us rewrite the longest-line program with \code{line}, \code{longest}, and \code{max} as external variables.
This requires changing the calls, declarations, and bodies of all three functions.

\begin{lstlisting}
	#include <stdio.h>
	#define MAXLINE 1000 	/* maximum input line size */

	int max; 				/* maximum length seen so far */
	char line[MAXLINE]; 	/* current input line */
	char longest[MAXLINE]; 	/* longest line saved here */

	int getline(void);
	void copy(void);

	/* print longest input line; specialized version */
	main()
	{
		int len;
		extern int max;
		extern char longest[];
		max = 0;
		while ((len = getline()) > 0)
			if (len > max) {
				max = len;
				copy();
			}
		if (max > 0) 	/* there was a line */
		printf("%s", longest);
		return 0;
	}

	/* getline: specialized version */
	int getline(void)
	{
		int c, i;
		extern char line[];
		for (i = 0; i < MAXLINE - 1 && (c=getchar()) != EOF && c != '\n'; ++i)
			line[i] = c;
		if (c == '\n') {
			line[i] = c;
			++i;
		}
		line[i] = '\0';
		return i;
	}

	/* copy: specialized version */
	void copy(void)
	{
		int i;
		extern char line[], longest[];
			i = 0;
		while ((longest[i] = line[i]) != '\0')
			++i;
	}
\end{lstlisting}
The external variables in \code{main}, \code{getline} and \code{copy} are defined by the first lines of the example above, which state their type and cause storage to be allocated for them.
Syntactically, external definitions are just like definitions of local variables, but since they occur outside of functions, the variables are external.
Before a function can use an external variable, the name of the variable must be made known to the function; the declaration is the same as before except for the added keyword extern.

In certain circumstances, the extern declaration can be omitted.
If the definition of the external variable occurs in the source file before its use in a particular function, then there is no need for an extern declaration in the function.
The extern declarations in \code{main}, \code{getline} and \code{copy} are thus redundant.
In fact, common practice is to place definitions of all external variables at the beginning of the source file, and then omit all extern declarations.

If the program is in several source files, and a variable is defined in \emph{file1} and used in \emph{file2} and \emph{file3}, then extern declarations are needed in \emph{file2} and \emph{file3} to connect the occurrences of the variable.
The usual practice is to collect extern declarations of variables and functions in a separate file, historically called a header, that is included by \code{\#include} at the front of each source file.
The suffix \code{.h} is conventional for header names.
The functions of the standard library, for example, are declared in headers like \code{<stdio.h>}.
This topic is discussed at length in Chapter 4, and the library itself in Chapter 7 and Appendix B.

Since the specialized versions of \code{getline} and \code{copy} have no arguments, logic would suggest that their prototypes at the beginning of the file should be \code{getline()} and \code{copy()}.
But for compatibility with older C programs the standard takes an empty list as an old-style declaration, and turns off all argument list checking; the word void must be used for an explicitly empty list.
We will discuss this further in Chapter 4.

You should note that we are using the words \emph{definition} and \emph{declaration} carefully when we refer to external variables in this section.
``Definition'' refers to the place where the variable is created or assigned storage; ``declaration'' refers to places where the nature of the variable is stated but no storage is allocated.

By the way, there is a tendency to make everything in sight an \code{extern} variable because it appears to simplify communications -- argument lists are short and variables are always there when you want them.
But external variables are always there even when you don't want them.
Relying too heavily on external variables is fraught with peril since it leads to programs whose data connections are not all obvious -- variables can be changed in unexpected and even inadvertent ways, and the program is hard to modify.
The second version of the longest-line program is inferior to the first, partly for these reasons, and partly because it destroys the generality of two useful functions by writing into them the names of the variables they manipulate.

At this point we have covered what might be called the conventional core of C.
With this handful of building blocks, it's possible to write useful programs of considerable size, and it would probably be a good idea if you paused long enough to do so.
These exercises suggest programs of somewhat greater complexity than the ones earlier in this chapter.
\newline

\begin{ExerciseList}
\Exercise Write a program \code{detab} that replaces tabs in the input with the proper number of blanks to space to the next tab stop. Assume a fixed set of tab stops, say every \code{n} columns. Should \code{n} be a variable or a symbolic parameter?
\Exercise Write a program \code{entab} that replaces strings of blanks by the minimum number of tabs and blanks to achieve the same spacing. Use the same tab stops as for \code{detab}. When either a tab or a single blank would suffice to reach a tab stop, which should be given preference?
\Exercise Write a program to ``fold'' long input lines into two or more shorter lines after the last non-blank character that occurs before the \code{n}-th column of input. Make sure your program does something intelligent with very long lines, and if there are no blanks or tabs before the specified column.
\Exercise Write a program to remove all comments from a C program. Don't forget to handle quoted strings and character constants properly. C comments don't nest.
\Exercise Write a program to check a C program for rudimentary syntax errors like unmatched parentheses, brackets and braces. Don't forget about quotes, both single and double, escape sequences, and comments. (This program is hard if you do it in full generality.)
\end{ExerciseList}


